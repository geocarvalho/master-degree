% Modelo de relatório no estilo artigo em duas colunas
%\documentclass[twocolumn]{article}
\documentclass{article}

\usepackage[portuguese]{babel}
\usepackage[utf8]{inputenc}
\usepackage{amsmath}
\usepackage{subcaption}
\usepackage{mathtools}
\usepackage{graphicx}
\usepackage{color}
\usepackage{authblk}
\usepackage{todonotes}
\usepackage[colorlinks,citecolor=red,urlcolor=blue,bookmarks=false,hypertexnames=true]{hyperref}

\newcommand{\eqname}[1]{\tag*{#1}}% Tag equation with name
\usepackage{multicol}
\usepackage{geometry}
% Tamanho das margens:
\geometry{
	a4paper,
	total={170mm,257mm},
	left=30mm,
	right=20mm,
	top=20mm,
}
%%%%%%%%%%%%%%%%%%%%%%%%%%%%%%%%%%%%%%%%%
% Bibliografia estilo ABNT. Se não tiver instalado, comente a linha abaixo.
\usepackage[alf, abnt-etal-list=0, abnt-emphasize=bf,abnt-last-names=bibtex, abnt-etal-text=it, abnt-etal-cite=2]{abntex2cite}
%%%%%%%%%%%%%%%%%%%%%%%%%%%%%%%%%%%%%%%%%

% Dados de identificação
\title{Subgroup mining applied to DNA methylation datasets}
\author{George Carvalho, André Câmara, Ricardo Prudêncio}
\affil{Universidade Federal de Pernambuco}

\begin{document}

\maketitle        

% Resumo de no máximo 200 palavras
\begin{abstract}

\color{black}
\end{abstract}

\section{Objetivo}

Dado um grupo de individíduos e atributos atrelados a esses indivíduos, estamos interessados em encontrar subgrupos populacionais que são estatisticamente mais interessantes \cite{wen2017cell,park2019finding}. Sendo os indivíduos pacientes com dados de metilação de DNA e suas propriedades os dados de expressão de metilação (variável continua) queremos encontrar subgrupos nesses dados que tenham alguma característica intrínseca e não necessariamente explícita que sejam estatísticamente significante. Fazendo isso é possível indicar possíveis biomarcadores para subgrupos específicos em uma população \cite{wen2017cell}.

\section{Metodologia}

Uma alternativa é classificar os dados de expressão de metilação em mais expresso ou menos expresso (binarizar) \cite{wang2019biomethyl, cedoz2018methylmix} para então encontrar subgrupos por meio \textit{subgroup mining}.

\begin{itemize}
    \item Coleta de dados
    \item Binarização dos dados de metilação. Ver como vai fazer isso (consensus?).
    
    \item Tratamento do atributo alvo (qual atributo e se vai ser binario, multiclasse ou contínuo)
    
    \item Descoberta de padroes
    \begin{enumerate}
        \item Opcao 1: aplicar diretamente o metodo de Tarcicio
        \item Opcao 2: Aplicar clustering seguido de aprendizagem de regras por cluster
    \end{enumerate}
\end{itemize}

\subsection{Processamento dos dados}

Baseado no estudo de \cite{jeschke2017cancers} foram selecionados dois datasets, GSE72245 e GSE72251 disponibilizados no GEO datasets, cada um contendo 118 e 119 amostras tumorais de cancêr de mama respectivamente. Cada paciente tem um subtipo molecular de câncer de mama: luminal A, luminal B, HER2 e basal-like. Esses subtipos estão associados ao prognóstico do paciente e é a nossa classe alvo em uma primeira análise.

O download dos dados foi feito usando a biblioteca GEOquery do Bioconductor, além disso foi feita uma filtragem de qualidade usando a biblioteca Minfi também do Bioconductor. Foram removidos CpGs com baixa qualidade de sinal e após essa primeira filtragem foi feita uma normalização por subconjunto de quartil dos dados visando minimizar variações entre amostras. Outro filtro foi a remoção de probes de baixa qualidade em pelo menos uma amostra, além de probes com SNPs comuns que podem afetar CpGs e probes que são conhecidas por alinhar em  múltiplas regiões do genoma que também foram removidas.

Após a filtragem, os valores beta de cada região foram obtidos usando a biblioteca Minfi. Como os valores beta ficam entre 0 e 1, eles foram categorizados em low (<=0.2), middle (>0.2, <0.8) e high (>=0.8), de acordo com o estudo de \cite{du2010beta} para serem usados como classe alvo no algoritmo de subgroup discovery ou de clustering.



\section{Formalização}


% Dado um banco de dados $D = (I, A)$ contendo indivíduos $I$ e seus atributos $A$, cada exemplo em $D,I = {a_1, ..., a_m}$ é um conjunto de valores correspondente aos atributos $A_1, ..., A_m$. A

Seja $\mathcal{D}$ um conjunto de $n$ indivíduos, descrito por um conjunto de $m$ atributos $ \mathcal{A} = \{A_1, ..., A_m\}$. Cada indivíduo $d_i \in \mathcal{D}$ é representado pelo vetor $\vec{a_i} = (a_{i1},\dots,a_{1m})$, onde $a_{1j} =  A_j(d_i)$. Um {\it subgrupo} é definido por um subconjunto de indivíduos em $\mathcal{D}$, representados por um subconjunto de atributos em $\mathcal{A}$ com valores em comum. A representação dos indivíduos de um subgrupo é chamada de {\it perfil}.  


Formalmente, um perfil $P_k$ é dado por um subconjunto de atributos $\mathcal{A}_k \subset \matcal{A}$ e uma conjunção de predicados com valores associados a esses atributos: $\bigwedge\limits_{A_j \in \mathcal{A}_k} (A_j = v_{kj})$. O subgrupo $S_{k}:=\{d_i \in \mathcal{D} | \forall A_j \in \mathcal{A}_k,  a_{ij} = v_{kj}\}$ é então o conjunto de todos os indivíduos com valores de atributos definidos pelo perfil $P_k$ \cite{park2019finding}. \textcolor{red}{DAR UM EXEMPLO}

%o i-ésimo valor de atributo do predicado, por exemplo sexo = "mulher", estilo = "fumante" e doença = "câncer". Um subgrupo $S_X:=\{I \in D | X(I)= true\}$ é o conjunto de todos os indivíduos que possuem o perfil $X$ \cite{park2019finding}.

Além disso é preciso selecionar um ou mais critérios (função de qualidade) para caracterizar o quanto determinado subgrupo é significativo naquela população. Algumas medidas comumentemente usadas é o tamanho do subgrupo e a média da variável de interesse no subgrupo \cite{park2019finding}.

%%%%%%%%%%%%%%%%%%%%%%%%%%
% BIBLIOGRAFIA 
% Estilo de bibliografia ABNT. Se não tiver instalado, mude para plain ou ieeetr

%\bibliographystyle{plain} % Inclua isso se não tiver ABNTEX instalado
\newpage
\bibliography{refs}
%\begin{thebibliography}{refs}
%\bibitem{}

%\end{thebibliography}
\end{document}
